%%% This Beamer example was created by LianTze Lim, April 2017.

%%%% This is a VERY simple and minimalistic beamer theme,
%%%% even reminiscent of marker pens on transparencies!
%%%% It mimics the look of the "seminar" package, which
%%%% can only be used with plain TeX.
%%%% There are also some comments and example to show how
%%%% to customise various elements, e.g. the font and colours.

\documentclass[12pt]{beamer}
%% If you'd like the default font size to be even larger, use 14pt or 17pt; these are supported by Beamer.

\graphicspath{{Figures/}{Figures}}

%%%%%%%%%%%%%%%%%%%%%%%%%%%%%%%%%%%%%%%%%
% These lines should usually go into a .sty file,
% but I'll leave them here so that it's easier to
% see how to customise a Beamer theme.
% Remember, the Beamer manual is your friend!!
% http://texdoc.net/pkg/beamer
%
%% So if your re-definitions have a @ somewhere, you
%% _MUST_ put a \makeatletter before these lines and then
%% \makeatother after them. This trick can only be done
%% in the preamble! BUT if you're doing these re-definitions
%% in a .sty file (so that you \usepackage it later), you
%% don't need the \makeatletter and \makeatother.
\makeatletter

%% Set the left and right margins
\setbeamersize{text margin left=1em,text margin right=1em}

%% FONTS
\setbeamerfont{title}{series=\bfseries,size=\LARGE}
\setbeamerfont{subtitle}{series=\bfseries,size=\Large}
\setbeamerfont{frametitle}{series=\bfseries,size=\small}
\setbeamerfont{block title}{series=\bfseries,size=\normalsize}
\setbeamerfont{footline}{size=\small}

%% COLOURS
%% If you'd like everything to have the same colour
\usebeamercolor{structure}
\setbeamercolor{normal text}{fg=structure.fg}

%% Add a line after the frametitle
\addtobeamertemplate{frametitle}{}{\vspace*{-1ex}\rule{\textwidth}{1pt}}

%% Use circular discs as itemized list markers;
%% there's an existing option in Beamer for it so I'll use it
\setbeamertemplate{itemize items}[circle]

%% Remove default navigation symbols (We'll add the ones we need in the footline
\setbeamertemplate{navigation symbols}{}


%% And before the footline... actually we'd like to re-define
%% the footline
\setbeamertemplate{footline}{%
   %% Beamer headlines and footlines are always full-paperwidth, so if you want the horizontal line to
   %% not span it entirely you'll need to do a bit of arithmetic
   \centering
   \begin{minipage}{\dimexpr\paperwidth-\beamer@leftmargin-\beamer@rightmargin\relax}
   \centering
   \rule{\linewidth}{1pt}\vskip2pt
   \usebeamerfont{footline}%
   \usebeamercolor{footline}%
   %% The frame number smack in the middle
   \hfill\insertframenumber/\inserttotalframenumber
   \hfill%
   %% ONLY the navigation symbols we want at the far right.
   %% We use an \llap so that it takes up zero width, and doesn't throw the page number off-centre!
   \llap{\insertframenavigationsymbol\insertbackfindforwardnavigationsymbol}\par
   \end{minipage}\vskip2pt
}

\setbeamercolor{block title alerted}{fg=white,bg=brown}

\makeatother
%%%% END STYLE CUSTOMISATION %%%%%%%%%%%%

\usepackage[english]{babel}
\usepackage[latin1]{inputenc}
\usepackage[T1]{fontenc}
\usepackage{graphicx}

\usepackage{subcaption}

\usepackage{times}

\usepackage{graphics}
%\usepackage[draft]{graphics}

\usepackage{xspace}
\usepackage{amsmath}
\usepackage{bm}
\usepackage{pgfpages}
\usepackage{fancybox}
\usepackage{threeparttable}
\usepackage{bbding}
\usepackage{verbatim}
\usepackage{booktabs}

\usepackage{enumitem}
\usepackage{pifont}

\usepackage{natbib}
\bibpunct{(}{)}{;}{a}{}{,}

\newcommand{\pkg}[1]{{\normalfont\fontseries{b}\selectfont #1}}
\let\proglang=\textsf
\let\code=\texttt

\newcommand{\btheta}{ \mbox{\boldmath $\theta$}}
\newcommand{\bbeta}{ \mbox{\boldmath $\beta$}}
\newcommand{\balpha}{ \mbox{\boldmath $\alpha$}}
\newcommand{\by}{ \mbox{\bf y}}
\newcommand{\bY}{ \mbox{\bf Y}}
\newcommand{\bX}{ \mbox{\bf X}}
\newcommand{\bH}{ \mbox{\bf H}}
\newcommand{\bI}{ \mbox{\bf I}}


\title{Introduction to spatial and spatio-temporal analysis}
\subtitle{}
\author{Bayesian modelling for spatial and spatio-temporal data}
\institute{MSc in Epidemiology}
\date{Week 6}


\begin{document}

\begin{frame}[t]
  \titlepage
\end{frame}


%%%%%%%%%%%%%%%%%%%%%%%%%%%%%%%%%%%%%%%%%%%%%%%%%%%%%%%%%%%%%%%%% Module Details
\begin{frame}
\frametitle{Goals for weeks 6-10}
\begin{itemize} [label=\ding{212}] \setlength\itemsep{\fill}
\item Building on the theoretical concepts and tools studied in weeks 1-5, we will now learn the relevant statistical methods used in spatial and spatio-temporal analysis and we will apply  them using computational tutorials and practicals/homeworks.
\item By the end of these five weeks, we will gain and master the fundamental strategies to:
\begin{itemize} [label=\ding{212}] \setlength\itemsep{\fill}
  \item \alert{manipulate}
  \item \alert{explore}
  \item \alert{visualize}
  \item \alert{model}
\end{itemize}
spatially and temporally-related data, and \alert{critically discuss and utilise the results}. 
%using freely-available and open-source tools and we will be confident in conducting analyses in a reproducible fashion.
%\vspace{10pt}
%\item We will handle model complexity within a Bayesian framework applying the concepts learnt in weeks 1-5, with statistical inference carried out using Markov Chain Monte Carlo (MCMC) methods.
\end{itemize}
\end{frame}

%%%%%%%%%%%%%%%%%%%%%%%%%%%%%%%%%%%%%%%%%%%%%%%%%%%%%%%%%%%%%%%%%

\begin{frame}
\frametitle{Topics}
\begin{itemize} \setlength\itemsep{\fill}
  \item \textbf{Week 6}: Introduction to (i) spatial data types and coordinate reference systems, (ii) spatial autocorrelation, and (iii) disease mapping
  \item \textbf{Week 7}: Areal data analysis and \emph{discrete} spatial models that belong to the family of \alert{Conditional Autoregressive (CAR) distributions}, also known as Gaussian Markov Random Fields (GMRF)
  %We will study the most popular spatial prior distributions in disease mapping models, which belong to the family of \alert{Conditional Autoregressive (CAR) distributions}, also known as Gaussian Markov Random Fields (GMRF)
  \item \textbf{Week 8}: Geostatistical data analysis and \emph{continuous} spatial models with \alert{Gaussian process based smoothers}
  \item \textbf{Week 9}: Temporal processes and spatial-temporal analysis with focus on areal data
  \item \textbf{Week 10}: Temporal processes and spatio-temporal analysis with focus on geostatistical data; datasets realise for mini-project reports
    \end{itemize}
\end{frame}


\begin{frame}
\frametitle{Nearby things tend to be more alike...}
\vspace{15pt}
Waldo Tobler's first law of geography:
\begin{quote}{}
  \centering \emph{Everything is related to everything else, but near things are more related than distant things \citep{Tobler1970}}
\end{quote}

\vspace{15pt}
\textbf{Interpretation}:\\
Data that are close together in space (or time) are often more similar  than those far apart
(\alert{except when they aren't!})
\end{frame}

%%%%%%%%%%%%%%%%%%%%%%%%%%%%%%%%%%%%%%%%%%%%%%%%%%%%%%%%%%%%%%%%
\begin{frame}
\frametitle{What is spatial analysis?}
Spatial analysis represents a collection of techniques and models that explicitly use the spatial referencing associated with each data value or object that is specified within the system under study \citep{Haining2003}.
\begin{scriptsize}
\begin{tabular}{cc}
\begin{minipage}{0.45\linewidth}
\begin{alertblock}{\small Non-Spatial Analysis}
\begin{itemize}
\item Spatial data are analyzed using conventional statistical methods
\item The geographical coordinates are excluded from the computational procedures
\item The results are independent of the spatial arrangement of the geographical entities
%\item Observations or entities are assumed to be independent and identically distributed.
\item The analysis can also include temporal dependence
\end{itemize}
\end{alertblock}
\end{minipage}

&
%\hspace{3pt}

\begin{minipage}{0.45\linewidth}
\begin{alertblock}{\small Spatial Analysis}
\begin{itemize}
\item Spatial data are analysed using spatial statistical methods
\item The geographical coordinates are included into the computational procedures
\item The results depend on the spatial arrangement of the geographical entities
\item The analysis can also include temporal dependence
\end{itemize}
\end{alertblock}
\end{minipage}
\end{tabular}
\end{scriptsize}
\end{frame}

\begin{frame}
\frametitle{Article and book cited in this lecture}
\footnotesize{
\bibliographystyle{asa}
\bibliography{refs}
}
\end{frame}

\end{document} 