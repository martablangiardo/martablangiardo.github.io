%%% This Beamer example was created by LianTze Lim, April 2017.

%%%% This is a VERY simple and minimalistic beamer theme,
%%%% even reminiscent of marker pens on transparencies!
%%%% It mimics the look of the "seminar" package, which
%%%% can only be used with plain TeX.
%%%% There are also some comments and example to show how
%%%% to customise various elements, e.g. the font and colours.

\documentclass[12pt]{beamer}
%% If you'd like the default font size to be even larger, use 14pt or 17pt; these are supported by Beamer.

\graphicspath{{Figures/}{Figures}}

%%%%%%%%%%%%%%%%%%%%%%%%%%%%%%%%%%%%%%%%%
% These lines should usually go into a .sty file,
% but I'll leave them here so that it's easier to
% see how to customise a Beamer theme.
% Remember, the Beamer manual is your friend!!
% http://texdoc.net/pkg/beamer
%
%% So if your re-definitions have a @ somewhere, you
%% _MUST_ put a \makeatletter before these lines and then
%% \makeatother after them. This trick can only be done
%% in the preamble! BUT if you're doing these re-definitions
%% in a .sty file (so that you \usepackage it later), you
%% don't need the \makeatletter and \makeatother.
\makeatletter

%% Set the left and right margins
\setbeamersize{text margin left=1em,text margin right=1em}

%% FONTS
\setbeamerfont{title}{series=\bfseries,size=\LARGE}
\setbeamerfont{subtitle}{series=\bfseries,size=\Large}
\setbeamerfont{frametitle}{series=\bfseries,size=\small}
\setbeamerfont{block title}{series=\bfseries,size=\normalsize}
\setbeamerfont{footline}{size=\small}

%% COLOURS
%% If you'd like everything to have the same colour
\usebeamercolor{structure}
\setbeamercolor{normal text}{fg=structure.fg}

%% Add a line after the frametitle
\addtobeamertemplate{frametitle}{}{\vspace*{-1ex}\rule{\textwidth}{1pt}}

%% Use circular discs as itemized list markers;
%% there's an existing option in Beamer for it so I'll use it
\setbeamertemplate{itemize items}[circle]

%% Remove default navigation symbols (We'll add the ones we need in the footline
\setbeamertemplate{navigation symbols}{}


%% And before the footline... actually we'd like to re-define
%% the footline
\setbeamertemplate{footline}{%
   %% Beamer headlines and footlines are always full-paperwidth, so if you want the horizontal line to
   %% not span it entirely you'll need to do a bit of arithmetic
   \centering
   \begin{minipage}{\dimexpr\paperwidth-\beamer@leftmargin-\beamer@rightmargin\relax}
   \centering
   \rule{\linewidth}{1pt}\vskip2pt
   \usebeamerfont{footline}%
   \usebeamercolor{footline}%
   %% The frame number smack in the middle
   \hfill\insertframenumber/\inserttotalframenumber
   \hfill%
   %% ONLY the navigation symbols we want at the far right.
   %% We use an \llap so that it takes up zero width, and doesn't throw the page number off-centre!
   \llap{\insertframenavigationsymbol\insertbackfindforwardnavigationsymbol}\par
   \end{minipage}\vskip2pt
}

\setbeamercolor{block title alerted}{fg=white,bg=brown}

\makeatother
%%%% END STYLE CUSTOMISATION %%%%%%%%%%%%

\usepackage[english]{babel}
\usepackage[latin1]{inputenc}
\usepackage[T1]{fontenc}
\usepackage{graphicx}

\usepackage{subcaption}

\usepackage{times}

\usepackage{graphics}
%\usepackage[draft]{graphics}

\usepackage{xspace}
\usepackage{amsmath}
\usepackage{bm}
\usepackage{pgfpages}
\usepackage{fancybox}
\usepackage{threeparttable}
\usepackage{bbding}
\usepackage{verbatim}
\usepackage{booktabs}

\usepackage{natbib}
\bibpunct{(}{)}{;}{a}{}{,}

\newcommand{\pkg}[1]{{\normalfont\fontseries{b}\selectfont #1}}
\let\proglang=\textsf
\let\code=\texttt

\newcommand{\btheta}{ \mbox{\boldmath $\theta$}}
\newcommand{\bbeta}{ \mbox{\boldmath $\beta$}}
\newcommand{\balpha}{ \mbox{\boldmath $\alpha$}}
\newcommand{\by}{ \mbox{\bf y}}
\newcommand{\bY}{ \mbox{\bf Y}}
\newcommand{\bX}{ \mbox{\bf X}}
\newcommand{\bH}{ \mbox{\bf H}}
\newcommand{\bI}{ \mbox{\bf I}}


\title{Global measures of spatial autocorrelation}
\subtitle{}
\author{Bayesian modelling for spatial and spatio-temporal data}
\institute{MSc in Epidemiology}
\date{Week 6}


\begin{document}

\begin{frame}[t]
  \titlepage
\end{frame}

\begin{frame}
\frametitle{Measuring spatial autocorrelation}
\begin{itemize}\setlength\itemsep{\fill}
%\item A number of approaches have been suggested for measuring spatial autocorrelation.
\item Global measures of spatial autocorrelation share a common structure: calculate the similarity of values at locations $i$ and $j$, then weight the similarity by the proximity of locations $i$ and $j$.
\item Null hypothesis: spatial randomness (independent observations).
\item Form of global measure of spatial autocorrelation:
$$T= c \frac{\sum_{i=1}^N \sum_{j=1}^N w_{ij} \times \hbox{similarity}_{ij}}{\sum_{i=1}^N \sum_{j=1}^N w_{ij}}$$
\end{itemize}
\footnotesize{
\begin{itemize}\setlength\itemsep{\fill}
\item $c$ constant
\item $N$ number of areas
\item $w_{ij}$ weights reflecting the proximity between areas $i$ and $j$
\item $\hbox{similarity}_{ij}$ measure of similarity between data values in areas $i$ and $j$
\end{itemize}
}
\end{frame}

\begin{frame}
\frametitle{Global Moran's I test for spatial autocorrelation}
\begin{itemize}\setlength\itemsep{\fill}
\item Moran's I test can be applied to the data directly, or to the residuals from a regression model (as shown in tutorial 6.1).
\item Let $\{Z_{i}: i,\dots,N\}$ represent spatially referenced data (or residuals) for $N$ spatial locations.
\item The global Moran's I statistic is:\\
$$ I  = \frac{N \sum_{i=1}^N \sum_{j=1}^N w_{ij}(Z_i - \overline{Z})(Z_j - \overline{Z})}
{\left(\sum_i\sum_j w_{ij}\right)\sum_{i=1}^N (Z_i - \overline{Z})^2} $$
\vspace{5pt}
where $\overline{Z}=(1/N)\sum_i^N Z_i$ is the spatial mean and $w_{ij}$ are the spatial weights
\end{itemize}
\end{frame}

\begin{frame}
\frametitle{Global Moran's I interpretation}
$I$ tell us if most pairs of neighbouring areas have the same sign regarding their deviation from the mean.
\begin{itemize}\setlength\itemsep{\fill}
    \item If there is no spatial dependence, $I$ will be close to 0 (i.e. spatial pattern is random)
    \item If $I > 0$ and significant, then areas close together (as defined by $w_{ij}$) will tend to have similar values (i.e. clustering of like value)
    \item If $I < 0$ and significant, clustering of dissimilar (i.e. alternating) values
    \item ``Significance'' will be done using Monte Carlo (MC) approach (i.e. the data are repeatedly randomly assigned to different areas, and the statistic calculated under each permutation, yielding a comparison distribution)
\end{itemize}
\end{frame}

\begin{frame}
\frametitle{References}
\begin{itemize}\setlength\itemsep{\fill}
\item Haining R., Guangquan L. (2020), Modelling Spatial and Spatial-Temporal Data. A Bayesian Approach, CRC Press, Section 6.2.4.2
\item Waller L. A., Gotway C.A. (2004), Applied Spatial Statistics for Public Health Data. John Wiley \& Sons,  Section 7.4
\end{itemize}
\end{frame}


\end{document} 